\documentclass[11pt,twoside]{article}
\usepackage{geometry}
\usepackage{enumerate}
\usepackage{latexsym,booktabs}
\usepackage{amsmath,amssymb}
\usepackage{graphicx}
\usepackage[singlespacing]{setspace}

\geometry{a4paper,left=2cm,right=2.0cm, top=2cm, bottom=2.0cm}

\newtheorem{Definition}{Definition}
\newtheorem{Theorem}{Theorem}
\newtheorem{Lemma}{Lemma}
\newtheorem{Corollary}{Corollary}
\newtheorem{Proposition}{Proposition}
\newtheorem{Algorithm}{Algorithm}
\numberwithin{Theorem}{section}
\numberwithin{Definition}{section}
\numberwithin{Lemma}{section}
\numberwithin{Algorithm}{section}
\numberwithin{equation}{section}

% include subsubsections on TOC
\setcounter{tocdepth}{3}
\setcounter{secnumdepth}{3}

% hyperlinks
\usepackage{hyperref}

% accronyms and definitions
\usepackage[acronym,automake,nonumberlist]{glossaries}

\makeglossaries

\newglossaryentry{latex}
{
        name=latex,
        description={Is a mark up language specially suited for 
scientific documents}
}

\newglossaryentry{maths}
{
        name=mathematics,
        description={Mathematics is what mathematicians do}
}

\newglossaryentry{formula}
{
        name=formula,
        description={A mathematical expression}
}

\newacronym{gcd}{GCD}{Greatest Common Divisor}

\newacronym{lcm}{LCM}{Least Common Multiple}


\begin{document}

\pagestyle{empty}

% =============================================================================
% Title page
% =============================================================================
\begin{titlepage}
\vspace*{.5em}
\center
\textbf{\large{The School of Mathematics}} \\
\vspace*{1em}
\begin{figure}[!h]
\centering
\includegraphics[width=180pt]{CentredLogoCMYK.jpg}
\end{figure}
\vspace{2em}
\textbf{\Huge{Incremental Learning\\of Recommender Systems}}\\[2em]
\textbf{\LARGE{by}}\\
\vspace{2em}
\textbf{\LARGE{Rodrigo LARA MOLINA}}\\
\vspace{6.5em}
\Large{Dissertation Presented for the Degree of\\
MSc in Statitics with Data Science}\\
\vspace{6.5em}
\Large{August 2019}\\
\vspace{3em}
% \Large{Supervised by\\Dr Very Important}
\Large{Supervisors: Sean MURPHY, Timos KORRES\\
Second Supervisors: Iain MURRAY, Victor ELVIRA}
\vfill
\end{titlepage}

\cleardoublepage

% =============================================================================
% Abstract, acknowledgments, and own work declaration
% =============================================================================
\begin{center}
\Large{Abstract}
\end{center}

Here comes your abstract ...

\clearpage

\begin{center}
\Large{Acknowledgments}
\end{center}

Here come your acknowledgments ...

\clearpage

\begin{center}
\Large{Own Work Declaration}
\end{center}

Here comes your own work declaration

\cleardoublepage



% =============================================================================
% Table of contents, tables, and pictures (if applicable)
% =============================================================================
\pagestyle{plain}
\setcounter{page}{1}
\pagenumbering{Roman}

\tableofcontents
\clearpage
\listoftables
\listoffigures
\printglossary[type=\acronymtype]
\printglossary[title=Special Terms]
\clearpage
\cleardoublepage

\pagenumbering{arabic}
\setcounter{page}{1}

\nocite{*}
\bibliographystyle{abbrv}
\clearpage

\section{Introduction}
\label{sec:intro}

Here I will write a very good, precise and brief introduction.
Particularly Section \ref{sec:background} is good!

The \Gls{latex} typesetting markup language is specially suitable 
for documents that include \gls{maths}. \Glspl{formula} are 
rendered properly an easily once one gets used to the commands.

Given a set of numbers, there are elementary methods to compute 
its \acrlong{gcd}, which is abbreviated \acrshort{gcd}. This 
process is similar to that used for the \acrfull{lcm}.

\subsection{Problem Statement}
\label{sec:problem}

Techniques even better because.
\begin{enumerate}
 \item They're magnificent.
 \item If they work.
\end{enumerate}

\subsection{Objectives}
\label{sec:objective}

Techniques even better because.
\begin{enumerate}
 \item They're magnificent.
 \item If they work.
\end{enumerate}

\subsection{Disseration Outline}
\label{sec:outline}

Techniques even better because.
\begin{enumerate}
 \item They're magnificent.
 \item If they work.
\end{enumerate}

\clearpage



\section{Background and Related Work}
\label{sec:background}

In the following, I explain some background stuff. I should really cut this short, but BlaBlaBlaBla BlaBlaBlaBlaBla Bla Bla BlaBlaBla Bla Bla BlaBlaBla Bla BlaBla BlaBla Bla BlaBlaBlaBla Bla BlaBla Bla Bla Bla BlaBla BlaBlaBlaBla BlaBlaBlaBlaBla Bla Bla BlaBlaBla Bla. Bla BlaBlaBla Bla BlaBla BlaBla Bla BlaBlaBlaBla Bla BlaBla Bla Bla Bla BlaBla BlaBlaBlaBla. BlaBlaBlaBlaBla Bla Bla BlaBlaBla Bla Bla BlaBlaBla Bla BlaBla BlaBla Bla BlaBlaBlaBla Bla BlaBla Bla Bla Bla BlaBla BlaBlaBlaBla BlaBlaBlaBlaBla Bla Bla BlaBlaBla Bla Bla BlaBlaBla Bla BlaBla BlaBla Bla BlaBlaBlaBla Bla BlaBla Bla Bla Bla BlaBla BlaBlaBlaBla BlaBlaBlaBlaBla Bla Bla BlaBlaBla Bla Bla BlaBlaBla Bla BlaBla BlaBla Bla. BlaBlaBlaBla Bla BlaBla Bla Bla Bla BlaBla BlaBlaBlaBla BlaBlaBlaBlaBla Bla Bla BlaBlaBla Bla Bla BlaBlaBla Bla BlaBla BlaBla Bla BlaBlaBlaBla Bla BlaBla Bla Bla Bla BlaBla.


But I can also end a line with a double backslash.

\clearpage

\subsection{Recommender Systems}
\label{sec:rec-sys}

Techniques even better because.
\begin{enumerate}
 \item They're magnificent.
 \item If they work.
\end{enumerate}

\subsubsection{Neural Recommender Systems}
\label{sec:nn-rs}

Techniques even better because.
\begin{enumerate}
 \item They're magnificent.
 \item If they work.
\end{enumerate}


\subsubsection{Performance Metrics for Binary Classification}
\label{sec:perf-met}

Techniques even better because.
\begin{enumerate}
 \item They're magnificent.
 \item If they work.
\end{enumerate}

\clearpage

\subsection{Incremental Learning}
\label{sec:inc-learn}

Models are \emph{very} helpful because.
\begin{itemize}
 \item They're good.
 \item They're helpful.
\end{itemize}

\subsubsection{Sample-based Incremental Learning}
\label{sec:sample-based}

Models are \emph{very} helpful because.
\begin{itemize}
 \item They're good.
 \item They're helpful.
\end{itemize}

\subsubsection{Model-based Incremental Learning}
\label{sec:model-based}

Models are \emph{very} helpful because.
\begin{itemize}
 \item They're good.
 \item They're helpful.
\end{itemize}

\subsubsection{Bayesian Incremental Learning}
\label{sec:bayes-il}

Models are \emph{very} helpful because.
\begin{itemize}
 \item They're good.
 \item They're helpful.
\end{itemize}

\clearpage

\subsection{Variational Inference}
\label{sec:var-inf}

Techniques even better because.
\begin{enumerate}
 \item They're magnificent.
 \item If they work.
\end{enumerate}

\subsubsection{Stochastic Variational Inference}
\label{sec:svi}

Models are \emph{very} helpful because.
\begin{itemize}
 \item They're good.
 \item They're helpful.
\end{itemize}

\clearpage

\section{Experiments}
\label{sec:exp}

Now it's getting very technical \ldots{} I will cite \cite{shiina,groewe2001}. I will also show my incredible $\alpha$, $\beta$ and $\gamma$ mathematics and do some other fancy stuff.

\subsection{Data}
\label{sec:data}

For example look at this
\begin{equation}\label{eqn:aProblem}
\min{}\sum_{s\in\mathcal{S}}Pr_{s}\left[\sum_{t=1}^{T}\left(
\sum_{g\in\mathcal{G}}\left(\alpha_{gts}C_{g}^{0}+
p_{gts}C_{g}^{1}+\left(p_{gts}\right)^{2}C_{g}^{2}\right)
+\sum_{g\in\mathcal{C}}\gamma_{gts}C_{g}^{s}\right)\right],
\end{equation}
and you will see that it has a little number on the side so that I can refer to it as equation (\ref{eqn:aProblem}). Now if I do this
\begin{eqnarray}
\sum_{i=1}^{n}k_{i}&=&20\label{eqn:one}\\
\sum_{j=20}^{m}\delta_{i}&\geq{}&\eta{}\notag
\end{eqnarray}
I can align two formulae and control which one has a number on the side. It is (\ref{eqn:one}). I can also do something like this
\begin{displaymath}
Y_{l}=\left[\begin{array}{cc}
             \left(y_{s}+i\frac{b_{c}}{2}\right)\frac{1}{\tau{}^{2}} &
             -y_{s}\frac{1}{\tau{}e^{-i\theta^{s}}}\\
             -y_{s}\frac{1}{\tau{}e^{i\theta^{s}}} &
             y_{s}+i\frac{b_{c}}{2}
             \end{array}\right],
\end{displaymath}
and it won't have a number on the side. Now if I have to do some huge mathematics I'd better structure it a little and include linebreaks etc. so that it fits on one page.
\begin{eqnarray}\label{eqn:horrible}
p_{l}^{f}&=&G_{l11}\left(2v_{F(l)}\bar{v}_{F(l)}-\bar{v}_{F(l)}^{2}\right)\\
&+&
\bar{v}_{F(l)}\bar{v}_{T(l)}
\left[
B_{l12}\sin{}(\bar{\delta{}}_{F(l)}-\bar{\delta{}}_{T(l)})
+G_{l12}\cos{}(\bar{\delta{}}_{F(l)}-\bar{\delta{}}_{T(l)})
\right]\notag\\
&+&
\left[\begin{array}{r}
      \bar{v}_{T(l)}
      \left[
      B_{l12}\sin{}(\bar{\delta{}}_{F(l)}-\bar{\delta{}}_{T(l)})
      +G_{l12}\cos{}(\bar{\delta{}}_{F(l)}-\bar{\delta{}}_{T(l)})
      \right]\\
      \bar{v}_{F(l)}
      \left[
      B_{l12}\sin{}(\bar{\delta{}}_{F(l)}-\bar{\delta{}}_{T(l)})
      +G_{l12}\cos{}(\bar{\delta{}}_{F(l)}-\bar{\delta{}}_{T(l)})
      \right]\\
      \bar{v}_{F(l)}\bar{v}_{T(l)}
      \left[
      B_{l12}\cos{}(\bar{\delta{}}_{F(l)}-\bar{\delta{}}_{T(l)})
      -G_{l12}\sin{}(\bar{\delta{}}_{F(l)}-\bar{\delta{}}_{T(l)})
      \right]\\
      \bar{v}_{F(l)}\bar{v}_{T(l)}
      \left[
      -B_{l12}\cos{}(\bar{\delta{}}_{F(l)}-\bar{\delta{}}_{T(l)})
      +G_{l12}\sin{}(\bar{\delta{}}_{F(l)}-\bar{\delta{}}_{T(l)})
      \right]\\
      \end{array}\right]
\cdot{}
\left[\begin{array}{c}
      v_{F(l)}-\bar{v}_{F(l)}\\
      v_{T(l)}-\bar{v}_{T(l)}\\
      \delta_{F(l)}-\bar{\delta{}}_{F(l)}\\
      \delta_{T(l)}-\bar{\delta{}}_{T(l)}
      \end{array}\right],\notag
\end{eqnarray}
This is a lot of fun!

\clearpage

\subsection{Base Model}
\label{sec:base-mod}

Finally we should have a nice picture like this one. However, I won't forget that figures and table are environments which float around in my document. So LaTeX will place them wherever it thinks they fit well with the surrounding text. I can try to change that with a float specifier, e.g. [!ht].
%This is a comment. The Compiler ignores it. It is here to remind me that, if I use a .jpeg or .png picture file as below I will need to compile the document with the pdflatex compiler.
\begin{figure}[!ht]
\centering
\includegraphics[width=0.5\textwidth]{scenTree.png}
\caption{Look at this scenario tree with funny times $t_{1}$ and scenarios $s_{1}$ etc.}
\label{fig:scenarioTree}
\end{figure}
Now I want to use one of my own environments. I want to define something.
\begin{Definition}
 I define
$$
\Gamma_{\eta}:=\sum_{i=1}^{n}\sum_{j=i}^{n}\xi{}(i,j)
$$
\end{Definition}
I definitely need some good tables, so I do this.
\begin{table}[!ht]
\centering
\begin{tabular}{|ll|rrrr|}
\hline
Case&Generators&Therm. Units&Lines&Peak load: [MW]&[MVar]\\
\hline\hline
6 bus&3 at 3 buses&2&11&210&210\\
9 bus&3 at 3 buses&3&9&315&115\\
24 bus&33 at 11 buses&26&38&2850&580\\
30 bus&6 at 6 buses&5&41&189.2&107.2\\
39 bus&10 at 10 buses&7&46&6254.2&1387.1\\
57 bus&7 at 7 buses&7&80&1250.8&336.4\\
\hline
\end{tabular}
\caption{Something that doesn't make sense.}
\label{tab:things}
\end{table}
I should really refer to Table \ref{tab:things}.

\subsection{Training Regimes}
\label{sec:train-reg}

\noindent
Let:
\begin{eqnarray*}
\Omega_0 & = & \{(x,y,z,f): \text{ satisfying } (9)-(19)\}, \\
\Omega_1 & = & \{(x,y,z,f): \text{ satisfying } (9),(11)-(20)\}, \\
\overline{\Omega}_0 & = & \{\textbf{0}\leq (x,y,z,f) \leq \textbf{1}: \text{ satisfying } (9)-(18)\}, \\
\overline{\Omega}_1 & = & \{\textbf{0}\leq (x,y,z,f) \leq \textbf{1}: \text{ satisfying } (9),(11)-(18),(20)\} \,.
\end{eqnarray*}
%
where $\textbf{0}$ and $\textbf{1}$ are vectors of appropriate dimensions with 0's and 1's, respectively.
Next we see that both $\Omega_0$ and $\Omega_1$ give equivalent formulations for the A-MSSP. In particular, the following statements hold:

\begin{Proposition}
$\Omega_0 \subseteq \Omega_1$.
\end{Proposition}

\noindent
\textbf{Proof.}
Let us suppose there exists $(x,y,z,f) \in \Omega_1$ such that $(x,y,z,f) \notin \Omega_0$.
Then, there exist indices $i \in I$ and $t \in \{0,\ldots,|T|-s_i\} $ with $x_i^t > \displaystyle 0.5\,\left( \sum_{h=1}^{s_i} x_i^{t+h} +1\right)$.
By definition, $x_i^t = 1$ and $x_i^{t+h} = 0$ for all $h \in \{1,\dots,s_i\}$. By~(11) and (12), $\displaystyle \sum_{h=1}^{s_i} f_i^{th}=1$, so $f_i^{th'}=1$ for some $h' \in \{1,\dots,s_i\}$.
But then,
\[ 0 \:=\: x_i^{t+h'} \:=\: \sum_{h=\max \{1, t+h'-(|T|-s_i)\}}^{\min\{s_i,t+h'\}} f_i^{t+h'-h,h} \:\ge\: f_i^{th'} \:=\: 1 \,,
\]
as $h' \in [\max \{1, t+h'-(|T|-s_i)\}, \min\{s_i,t+h'\}]$.
\hfill $\square$
\bigskip

\noindent
This immediately gives us
\begin{Corollary}
AS is a valid formulation for the A-MSSP.
\end{Corollary}

\noindent
Next we compare the Linear Programming (LP) relaxations of the two formulations.

\begin{Proposition}
$\overline{\Omega}_1 \subseteq  \overline{\Omega}_0 $.
\end{Proposition}

\noindent
\textbf{Proof.}
Homework
\hfill $\square$

\subsubsection{Baselines}
\label{sec:baselines}

Models are \emph{very} helpful because.
\begin{itemize}
 \item They're good.
 \item They're helpful.
\end{itemize}

\subsubsection{Bayesian Incremental Update}
\label{sec:biu}

Models are \emph{very} helpful because.
\begin{itemize}
 \item They're good.
 \item They're helpful.
\end{itemize}

\subsection{Results and Discussion}
\label{sec:res}

Techniques even better because.
\begin{enumerate}
 \item They're magnificent.
 \item If they work.
\end{enumerate}

\cleardoublepage

\section{Conclusions and Future Work}
\subsection{Conclusions}
\label{sec:conc}
I have no idea how to conclude, so I don't write much. But the stuff that follows is important.

\subsection{Future work}
\label{sec:fut-work}
I have no idea how to conclude, so I don't write much. But the stuff that follows is important.
\clearpage

%the entries have to be in the file literature.bib
\bibliography{literature}
\clearpage

\appendix
\section*{Appendices}
\addcontentsline{toc}{section}{Appendices}

\section{An Appendix}
\label{app:one}

Some stuff.
\clearpage

\section{Another Appendix}
\label{app:two}

Some other stuff.

\end{document}
